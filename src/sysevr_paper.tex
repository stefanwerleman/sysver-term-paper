% CVPR 2023 Paper Template
% based on the CVPR template provided by Ming-Ming Cheng (https://github.com/MCG-NKU/CVPR_Template)
% modified and extended by Stefan Roth (stefan.roth@NOSPAMtu-darmstadt.de)

\documentclass[12pt,twocolumn,letterpaper]{article}

%%%%%%%%% PAPER TYPE  - PLEASE UPDATE FOR FINAL VERSION
% \usepackage[review]{cvpr}      % To produce the REVIEW version
\usepackage{cvpr}              % To produce the CAMERA-READY version
%\usepackage[pagenumbers]{cvpr} % To force page numbers, e.g. for an arXiv version

% Include other packages here, before hyperref.
\usepackage{graphicx}
\usepackage{amsmath}
\usepackage{amssymb}
\usepackage{booktabs}
\usepackage[none]{hyphenat}
\usepackage{array}


% It is strongly recommended to use hyperref, especially for the review version.
% hyperref with option pagebackref eases the reviewers' job.
% Please disable hyperref *only* if you encounter grave issues, e.g. with the
% file validation for the camera-ready version.
%
% If you comment hyperref and then uncomment it, you should delete
% ReviewTempalte.aux before re-running LaTeX.
% (Or just hit 'q' on the first LaTeX run, let it finish, and you
%  should be clear).
\usepackage[pagebackref,breaklinks,colorlinks]{hyperref}


% Support for easy cross-referencing
\usepackage[capitalize]{cleveref}
\crefname{section}{Sec.}{Secs.}
\Crefname{section}{Section}{Sections}
\Crefname{table}{Table}{Tables}
\crefname{table}{Tab.}{Tabs.}


%%%%%%%%% PAPER ID  - PLEASE UPDATE
\def\cvprPaperID{*****} % *** Enter the CVPR Paper ID here
\def\confName{CVPR}
\def\confYear{2023}


\begin{document}

% <><><><><><><><><><><><><><><><><><><>
%               TITLE
% <><><><><><><><><><><><><><><><><><><>
\title{SySeVR: A Framework and Pathway for Using Deep Learning to Detect Software Vulnerabilities}


% <><><><><><><><><><><><><><><><><><><>
%              AUTHORS
% <><><><><><><><><><><><><><><><><><><>
\author{
    Carter Yanac\\
    University of Central Florida\\
    4000 Central Florida Blvd, Orlando, FL 32816\\
    {\tt\small yanac7562@knights.ucf.edu}
    \and
    Stefan Werleman\\
    University of Central Florida\\
    4000 Central Florida Blvd, Orlando, FL 32816\\
    {\tt\small stefanwerleman@knights.ucf.edu}
}
\maketitle

% <><><><><><><><><><><><><><><><><><><>
%           ABSTRACT - Stefan
% <><><><><><><><><><><><><><><><><><><>
\begin{abstract}
    Deep learning has been a catalyst for the latest breakthroughs in technological history. It has been 
    used in computer vision, finance, social media, and more. However, the latest development in deep 
    learning has not been under any of these categories. Researchers have developement deep learning 
    frameworks in the cybersecurity sector. More specifically there have been recent developments for 
    for analyzing software vulnerabilities as a primary aid for software engineers that work with sensitive 
    data and systems. This work paper covers a special type of deep learning framework called SySeVR which is 
    detects software vulnerabities in C/C++ code implementation. We are not only going to cover this framework, 
    in detail but we are also going to cover other existing related frameworks and how it enabled more 
    developments in deep learning such as deep representation learning, VulDeBERT, VulDeePecker, and SAVIOR.
\end{abstract}

% <><><><><><><><><><><><><><><><><><><>
%         INTRODUCTION - Stefan
% <><><><><><><><><><><><><><><><><><><>
\section{Introduction}
\label{sec:intro}

\cite{Lin20}

% <><><><><><><><><><><><><><><><><><><>
%         DEEP LEARNING - Stefan
% <><><><><><><><><><><><><><><><><><><>
\section{Deep Learning}
\label{sec:deep-learning}

\subsection{Recurrent Neural Network}
\label{sub:recurrent-neural-network}

\subsection{Convolutional Neural Network}
\label{sub:convolutional-neural-network}

\subsection{Deep Belief Network}
\label{sub:deep-belief-network}

% <><><><><><><><><><><><><><><><><><><>
%          SYSEVR FRAMEWORK
% <><><><><><><><><><><><><><><><><><><>
\section{SySeVR Framework}
\label{sec:sysevr-framework}

\cite{Li22}

% <><><><><><><><><><><><><><><><><><><>
%         ADDITIONAL FRAMEWORKS
% <><><><><><><><><><><><><><><><><><><>
\section{Additional Frameworks}
\label{sec:additional-frameworks}

\subsection{Automated Vulnerability Detection Machine Using Deep Representation Learning}
\label{sub:automated-vulnerability-detection-machine-using-deep-representation-learning}

\cite{Russell18}

\subsection{VulDeBERT: A Vulnerability Detection System Using BERT}
\label{sub:vuldebert}

\cite{Kim22}

\subsection{SAVIOR: Towards Bug-Driven Hybrid Testing}
\label{sub:savior}

\cite{Chen20}

\subsection{VulDeePecker: A Deep Learning-Based System for Multiclass Vulnerability Detection}
\label{sub:vuledeepecker}

\cite{Zou21}

% <><><><><><><><><><><><><><><><><><><>
%         FUTURE WORK - Carter
% <><><><><><><><><><><><><><><><><><><>
\section{Future Work}
\label{sec:future-work}

\subsection{Working with Version Control}
\label{sub:working-with-version-control}

\subsection{Integrating with CI/CD Pipelines}
\label{sub:integrating-with-cicd-pipelines}

% <><><><><><><><><><><><><><><><><><><>
%          CONCLUSION - Carter
% <><><><><><><><><><><><><><><><><><><>
\section{Conclusion}
\label{sec:conclusion}

% <><><><><><><><><><><><><><><><><><><>
%            REFERENCES
% <><><><><><><><><><><><><><><><><><><>
{\small
\bibliographystyle{ieee_fullname}
\bibliography{egbib}
}

\end{document}
